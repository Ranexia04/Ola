\documentclass[a4paper, 2pt]{report}

\usepackage[a4paper, total={6in, 9in}]{geometry}
\usepackage[table,xcdraw]{xcolor}
\usepackage{indentfirst}
\usepackage{mathtools}
\usepackage{graphicx}
\usepackage{float}
\usepackage{hyperref}

\renewcommand{\chaptername}{Parte}
\renewcommand{\contentsname}{Índice}
\renewcommand{\figurename}{Figura}
\renewcommand{\tablename}{Tabela}

\begin{document}
    

\begin{titlepage}
    \begin{center}
        \vspace*{3cm}
 
        \LARGE
        \textbf{Instituto Superior Técnico}
        \vskip 0.4cm
 
        \Large{MEEC}
        \vskip 0.2cm

        \Large{Instrumentação e Medidas}
        \vskip 3cm
        

 
        \Huge{\textbf{1º Trabalho de Laboratório}}
        \vskip 0.5cm

        \huge{\textbf{Amplificador Operacional}}
        \vskip 0.5cm

 
        \vfill
 
        \large
        \textbf{Grupo Nº 44}\\
        \vspace{0.3cm}
        Alexandre Rodrigues, 90002\\
        Henrique Lourenço, \\
        Rodrigo Aires, \\
        \vspace{1cm}

        \textbf{Turno:} 4ªf 17h30

    \end{center}
\end{titlepage}

\chapter{Amplificador Operacional em Montagem Inversora}

\par Texto sobre a figura \ref{figura_exemplo}.

\begin{figure}[H]
    \centering
    %\includegraphics[width = 4in]{}
    \caption{Legenda da figura}
    \label{figura_exemplo}
\end{figure}

\chapter{Amplificador Operacional em Montagem Integradora}



\chapter{Amplificador Operacional em Montagem Diferença}

\par Realiza-se uma montagem diferença com os parâmetros descritos no enunciado. O ampop é alimentado com \(\pm 15V\), como sugerido no \textit{datasheet} do dispositivo (OP07D).

\par De modo a estimar o ganho comum da montagem, calcula-se o ponto de funcionamento em repouso do circuito. Coloca-se uma tensão de \(13V\) comum aos dois terminais de entrada (condições idênticas às descritas no \textit{datasheet} do dispositivo para testar esta grandeza) e observa-se a tensão de saída. O ganho comum é dado por (\ref{3_1}):

\begin{equation}
    G_C = 20 \log_{10}(\frac{u_S}{13}) = -46.545773\text{dB}
    \label{3_1}
\end{equation}

\par De seguida, é realizada uma análise da resposta em frequência do circuito. É colocada uma fonte AC de \(250\text{mV}\) de amplitude na entrada \(u_A\), e uma fonte idêntica em oposição de fase em \(u_B\). Sabendo que o ganho diferencial é dado por (\ref{3_2}):

\begin{equation}
    G_D = \frac{u_S}{u_B - u_A}
    \label{3_2}
\end{equation}

\par Obtém-se o seguinte diagrama de bode \ref{3_bode}:

\begin{figure}[H]
    \centering
    \includegraphics[width=5in]{3_freq_bode.png}
    \caption{Ganho diferencial em função da frequência da montagem diferença}
    \label{3_bode}
\end{figure}

\par O CMRR do circuito vem da relação entre o ganho diferencial e o ganho comum, obtendo-se então (\ref{3_3}):

\begin{equation}
    \textit{CMRR} = G_D[\text{dB}] - G_C[\text{dB}] = 26.602118 + 46.545773 = 73.147891 \text{dB}
    \label{3_3}
\end{equation}

\par De modo a exemplificar a análise anterior, simula-se a reposta temporal do circuito nas condições idênticas à análise em frequência, escolhendo-se um sinal de \(1 \text{kHz}\) para o efeito. A resposta do circuito está representada na figura \ref{3_wave}:

\begin{figure}[H]
    \centering
    \includegraphics[width = 5in]{3_waveform.png}
    \caption{Resposta temporal da montagem diferença a um sinal diferencial sinusoidal}
    \label{3_wave}
\end{figure}



\chapter{Amplificador de Instrumentação}


\par Realiza-se a montagem referida no enunciado, com uma resistência de ganho \(R_G = 180\Omega\). O amplificador de instrumentação é alimentado com \(\pm 15V\), a tensão de alimentação máxima referida no \textit{datasheet} do dispositivo (LT1168).

\par De modo a estimar o ganho comum da montagem, calcula-se o ponto de funcionamento em repouso do circuito. Coloca-se uma tensão de \(10V\) comum aos dois terminais de entrada (condições descritas no \textit{datasheet} do dispositivo para testar o CMRR) e observa-se a tensão de saída. O ganho comum é dado por (\ref{4_1}):

\begin{equation}
    G_C = 20 \log_{10}(\frac{u_S}{10}) = -186.75523\text{dB}
    \label{4_1}
\end{equation}

\par De seguida, é realizada uma análise da resposta em frequência do circuito. É colocada uma fonte AC de \(20\text{mV}\) de amplitude na entrada \(u_A\), e uma fonte idêntica em oposição de fase em \(u_B\). Sabendo que o ganho diferencial é dado por (\ref{4_2}):

\begin{equation}
    G_D = \frac{u_S}{u_B - u_A}
    \label{4_2}
\end{equation}

\par Obtém-se o seguinte diagrama de bode \ref{4_bode}:

\begin{figure}[H]
    \centering
    \includegraphics[width=5in]{4_freq_bode.png}
    \caption{Ganho diferencial em função da frequência do amplificador de instrumentação}
    \label{4_bode}
\end{figure}

\par O CMRR do circuito vem da relação entre o ganho diferencial e o ganho comum, obtendo-se então (\ref{4_3}):

\begin{equation}
    \textit{CMRR} = G_D[\text{dB}] - G_C[\text{dB}] = 48.798155 + 186.75523 = 235.553385 \text{dB}
    \label{4_3}
\end{equation}

\par De modo a exemplificar a análise anterior, simula-se a reposta temporal do circuito nas condições idênticas à análise em frequência, escolhendo-se um sinal de \(1 \text{kHz}\) para o efeito. A resposta do circuito está representada na figura \ref{3_wave}:

\begin{figure}[H]
    \centering
    \includegraphics[width = 5in]{4_waveform.png}
    \caption{Resposta temporal da monta a um sinal diferencial sinusoidal}
    \label{3_wave}
\end{figure}



\chapter{Comparador}
\chapter{Medição e Amplificação de Tensões}

\huge{\textbf{Referências}} \newline
\newline
\normalsize
\href{https://www.ti.com/lit/ds/symlink/op07d.pdf}{OP07D} \newline
\href{https://www.analog.com/media/en/technical-documentation/data-sheets/1168fa.pdf}{LT1168} \newline

\end{document}